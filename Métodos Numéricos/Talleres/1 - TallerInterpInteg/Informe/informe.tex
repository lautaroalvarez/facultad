\documentclass[a4paper,10pt]{article}
% Idioma
\usepackage[spanish]{babel}
% La geometría del documento
\usepackage[paper=a4paper, hmargin=1.5cm, bottom=1.5cm, top=1.5cm]{geometry}
% Codificacion utf-8
\usepackage[utf8]{inputenc}
% Hyperlinking
\usepackage[colorlinks=true, linkcolor=black]{hyperref}
\usepackage{mathtools}

% Fancyhdr
\usepackage{fancyhdr}
\usepackage{fancyvrb}
% Para el número de la última página
\usepackage{lastpage}

\pagestyle{fancy}
\thispagestyle{fancy}
\fancyhf{}
%\lhead{Taller de Interpolación e Integración}
%\rhead{Métodos Numéricos}
\renewcommand{\footrulewidth}{0.4pt}
\renewcommand{\headrulewidth}{0pt}
\rfoot{\thepage /\pageref{LastPage}}
\lfoot{\small{ Lautaro Leonel Alvarez - Lib Nro 268/14 }}
% \paragraph y \subparagraph
\setcounter{secnumdepth}{5}
\setcounter{tocdepth}{5}

\usepackage{scrextend}

\begin{document}


\begin{addmargin}[8cm]{0cm}
	\textbf{Taller de Interpolación e Integración} \\
	Lautaro Leonel Alvarez \\
	Libreta Nro 268/14 \\
	Primer Cuatrimestre del 2016 \\
\end{addmargin}



\section{Ejercicio 1}
\par Veamos
\begin{equation}
	E_L^1(x) =  \frac{f^2(\xi(x))(x - x_j)(x - x_{j+1})}{2!}
\end{equation}
\par Pero por el enunciado sabíamos que
\begin{equation}
	f^2(x) \leq (1000 \frac{km}{h^2}, 1000 \frac{km}{h^2}) \forall x \in \mathds{R}
\end{equation}
\par Entonces
\begin{equation}
	E_L^1(x) \leq \frac{(1000 \frac{km}{h^2}, 1000 \frac{km}{h^2})(x - x_j)(x - x_{j+1})}{2}
\end{equation}
\par x es un punto entre $x_j$ y $x_{j+1}$, entonces $(x - x_j) \le \Delta t$ y $(x - x_{j+1}) \le \Delta t$.
\par Entonces nos queda
\begin{equation}
	(x - x_j)(x - x_{j+1}) \leq (\Delta t_i)^2
\end{equation}
\par Volviendo nos queda que
\begin{equation}
	E_L^1(x) \leq \frac{(1000 \frac{km}{h^2}, 1000 \frac{km}{h^2})(x - x_j)(x - x_{j+1})}{2} \leq \frac{(1000 \frac{km}{h^2}, 1000 \frac{km}{h^2})(\Delta t_i)^2}{2}
\end{equation}
\begin{equation}
	E_L^1(x) \leq \frac{(1000 \frac{km}{h^2}, 1000 \frac{km}{h^2})(\Delta t_i)^2}{2}  = (\frac{1000 \frac{km}{h^2} * (\Delta t_i)^2}{2}, \frac{1000 \frac{km}{h^2} * (\Delta t_i)^2}{2})
\end{equation}
\par Queremos que el error sea menor que $10^{-3}km$. Para esto vamos a plantear la ecuación de la circunferencia con $radio = 10^{-3}km$ y veremos que valores de x e y nos sirven para manternernos dentro de la circunferencia.

\begin{equation}
	x^2 + y^2 \leq (10^{-3} km)^2
\end{equation}
\par Vamos a tomar como x e y los valores obtenidos de $E_L^1(t)$) y buscaremos que cumplan la ecuación planteada.
\begin{equation}
	(\frac{1000 \frac{km}{h^2} * (\Delta t_i)^2}{2})^2 + (\frac{1000 \frac{km}{h^2} * (\Delta t_i)^2}{2})^2 \leq (10^{-3} km)^2
\end{equation}
\begin{equation}
	\Leftrightarrow \frac{1000000 \frac{km^2}{h^4} * (\Delta t_i)^4}{4} + \frac{1000000 \frac{km^2}{h^4} * (\Delta t_i)^4}{4} \leq 10^{-6} km^2
\end{equation}
\begin{equation}
	\Leftrightarrow \frac{1000000 \frac{km^2}{h^4} * (\Delta t_i)^4}{2} \leq 10^{-6} km^2
\end{equation}
\begin{equation}
	\Leftrightarrow (\Delta t_i)^4 \leq \frac{10^{-6} km^2 * 2}{1000000 \frac{km^2}{h^4}}
\end{equation}
\begin{equation}
	\Leftrightarrow (\Delta t_i)^4 \leq 20^{-12} h^4
\end{equation}
\begin{equation}
	\Leftrightarrow | \Delta t_i | \leq 0.001 h
\end{equation}
\par Sabemos entonces que si tomamos intervalos $\Delta t$ menores o iguale que 0.001 h estaremos respetando la cota de error impuesta. Tomaremos entonces $\Delta t$ = 0.001 h para tener el valor mas grande.



\section{Ejercicio 3}
No se bien como es la onda. La cota no es muy ajustada, da 3.1722e-5 para x y 1.3878e-20 para y. Pero aplicando en sqrt(x2 + y2) <= 10-3km me queda 3.1722e-5.


\section{Ejercicio 4}
\textbf{Mareado}:\\
Lineal: x -> 0.0073496  ;  y -> 0
Spline: x -> 6.4157e-04  ;  y -> 0

\textbf{Kane}:\\
Lineal: x -> 5.6843e-17  ;  y -> 1.1369e-16
Spline: x -> 0.017262  ;  y -> 0.017262



\section{Ejercicio 9}
\subsection{Lagrange}
\par Planteamos la ecuaci´on de Lagrange:
\begin{equation}
	P_L(x) = y_k \Sigma_{k=0}^n \frac{(x-x_o)...(x-x_{k-1})(x-x_{k+1})...(x-x_n)}{(x_k - x_0)...(x_k-x_{k-1})(x_k-x_{k+1})...(x_k-x_n)}
\end{equation}
\par Reemplazando por los valores dados nos queda:
\begin{equation}
	P_L(x) = \frac{(x-2)(x-4)(x-5)}{(1-2)(1-4)(1-5)} * 0 + \frac{(x-1)(x-4)(x-5)}{(2-1)(2-4)(2-5)} * 2 + \frac{(x-1)(x-2)(x-5)}{(4-1)(4-2)(4-5)} * 12 + \frac{(x-1)(x-2)(x-4)}{(5-1)(5-2)(5-4)} * 20
\end{equation}
\begin{equation}
	P_L(x) = (x-1) \left(\frac{(x^2-5x-4x+20)2}{6} + \frac{(x^2-5x-2x+10)12}{-6} + \frac{(x^2-4x-2x+2)20}{12}\right)
\end{equation}
\begin{equation}
	P_L(x) = (x-1)(0x^2 + 1x + 0)
\end{equation}
\par Nos queda entonces:
\begin{equation}
	P_L(x) = x^2 - x
\end{equation}

\subsection{Diferencias Divididas}

\begin{equation}
\begin{split}
	f[x_o] = f(x_0) = f(1) = 0 \\
	f[x_1] = f(x_1) = f(2) = 2 \\
	f[x_2] = f(x_2) = f(4) = 12 \\
	f[x_3] = f(x_3) = f(5) = 20
\end{split}
\end{equation}
\begin{equation}
\begin{split}
	f[x_o, x_1] = \frac{f[x_1] - f[x_0]}{x_1-x_0} = \frac{2-0}{2-1} = 2 \\
	f[x_1, x_2] = \frac{f[x_2] - f[x_1]}{x_2-x_1} = \frac{12-2}{4-2} = 3 \\
	f[x_2, x_3] = \frac{f[x_3] - f[x_2]}{x_3-x_2} = \frac{20-12}{5-4} = 8
\end{split}
\end{equation}
\begin{equation}
\begin{split}
	f[x_o, x_1, x_2] = \frac{f[x_2,x_1] - f[x_1,x_0]}{x_2-x_0} = \frac{3-2}{4-1} = \frac{1}{3} \\
	f[x_1, x_2, x_3] = \frac{f[x_3,x_2] - f[x_2,x_1]}{x_3-x_1} = \frac{8-3}{5-2} = \frac{5}{3}
\end{split}
\end{equation}
\begin{equation}
\begin{split}
	f[x_o, x_1, x_2, x_3] = \frac{f[x_3,x_2,x_1] - f[x_2,x_1,x_0]}{x_3-x_0} = \frac{\frac{5}{3}-\frac{1}{3}}{5-0} = \frac{4}{15}
\end{split}
\end{equation}
\par Por lo que me queda:
\begin{equation}
	P_3(x) = f[x_0] + f[x_0,x_1](x-x_0) + f[x_0,x_1,x_2](x-x_0)(x-x_1) + f[x_0,x_1,x_2,x_3](x-x_0)(x-x_1)(x-x_2)
\end{equation}
\begin{equation}
	P_3(x) = 0 + 2(x-1) + \frac{1}{3}(x-1)(x-2) + \frac{4}{15}(x-1)(x-2)(x-4)
\end{equation}


\section{Ejercicio 10}



\end{document}