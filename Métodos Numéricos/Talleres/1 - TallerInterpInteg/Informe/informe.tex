\documentclass[a4paper,10pt]{article}
% Idioma
\usepackage[spanish]{babel}
% La geometría del documento
\usepackage[paper=a4paper, hmargin=1.5cm, bottom=1.5cm, top=1.5cm]{geometry}
% Codificacion utf-8
\usepackage[utf8]{inputenc}
% Hyperlinking
\usepackage[colorlinks=true, linkcolor=black]{hyperref}
\usepackage{mathtools}

% Fancyhdr
\usepackage{fancyhdr}
\usepackage{fancyvrb}
% Para el número de la última página
\usepackage{lastpage}

\pagestyle{fancy}
\thispagestyle{fancy}
\fancyhf{}
%\lhead{Taller de Interpolación e Integración}
%\rhead{Métodos Numéricos}
\renewcommand{\footrulewidth}{0.4pt}
\renewcommand{\headrulewidth}{0pt}
\rfoot{\thepage /\pageref{LastPage}}
\lfoot{\small{ Lautaro Leonel Alvarez - Lib Nro 268/14 }}
% \paragraph y \subparagraph
\setcounter{secnumdepth}{5}
\setcounter{tocdepth}{5}

\usepackage{scrextend}

\begin{document}


\begin{addmargin}[8cm]{0cm}
	\textbf{Taller de Interpolación e Integración} \\
	Lautaro Leonel Alvarez \\
	Libreta Nro 268/14 \\
	Primer Cuatrimestre del 2016 \\
\end{addmargin}



\section{Ejercicio 1}
\par Como tenemos que utilizar interpolación fragmentaria lineal vamos a utilizar polinomios de Lagrange de grado 1 para cada fragmento que une $f(t_i)$ con $f(t_{i+1})$ \forall $i=0..n-1$. \\
\par Por la fórmula del error de los polinomios de Lagrange sabemos que: \\
\begin{equation}
	f^1(x) = P_L^1(x) + E_L^1(x)\ con\ E_L^1(x) = \frac{f^2(\xi(x))(x - x_j)(x - x_{j+1})}{2!}\ \forall i=0..n-1
\end{equation}
\par Queremos acotar:
\begin{equation}
	| E_L^1(x) | \leq 10^{-3} km
\end{equation}
\begin{equation}
	\Leftrightarrow | \frac{f^2(\xi(x))(x - x_j)(x - x_{j+1})}{2!} | \leq 10^{-3} km
\end{equation}
\par Pero por el enunciado sabíamos que
\begin{equation}
	f^2(x) \leq 1000 \frac{km}{h} \forall x \in \mathds{R}
\end{equation}
\par Entonces
\begin{equation}
	| \frac{f^2(\xi(x))(x - x_j)(x - x_{j+1})}{2} | \leq | \frac{(1000 km)(x - x_j)(x - x_{j+1})}{2!} |
\end{equation}
\par Veamos que se cumpla
\begin{equation}
	| \frac{(1000 km)(x - x_j)(x - x_{j+1})}{2} | \leq 10^{-3} km
\end{equation}
\begin{equation}
	\Leftrightarrow (1000 \frac{km}{h^2}) | \frac{(x - x_j)(x - x_{j+1})}{2} | \leq 10^{-3} km
\end{equation}
\begin{equation}
	\Leftrightarrow | \frac{(x - x_j)(x - x_{j+1})}{2} | \leq \frac{10^{-3} km}{1000 \frac{km}{h^2}}
\end{equation}
\begin{equation}
	\Leftrightarrow | \frac{(x - x_j)(x - x_{j+1})}{2} | \leq 10^{-6} h^2
\end{equation}
\begin{equation}
	\Leftrightarrow |(x - x_j)(x - x_{j+1}) | \leq 20^{-6} h^2
\end{equation}
\par x es un punto entre $x_j$ y $x_{j+1}$, entonces $(x - x_j) \le \Delta t$ y $(x - x_{j+1}) \le \Delta t$.
\par Entonces nos queda
\begin{equation}
	|(x - x_j)(x - x_{j+1}) | \leq | (\Delta t_i)^2 |
\end{equation}
\par Veamos entonces si
\begin{equation}
	| (\Delta t_i)^2 | \leq 20^{-6} h^2
\end{equation}
\begin{equation}
	\Leftrightarrow  \Delta t_i \leq 0.000125 h
\end{equation}

\par Volviendo tenemos que
\begin{equation}
	\Delta t_i \leq 0.000125 h
\end{equation}
\begin{equation}
	 \Rightarrow | \frac{(x - x_j)(x - x_{j+1})}{2} | \leq \frac{10^{-3} km}{1000 \frac{km}{h^2}}
\end{equation}
\begin{equation}
	 \Rightarrow | \frac{(1000 km)(x - x_j)(x - x_{j+1})}{2} | \leq 10^{-3} km
\end{equation}
\par Entonces
\begin{equation}
	| E_L^1(x) | = | \frac{f^2(\xi(x))(x - x_j)(x - x_{j+1})}{2} | \leq | \frac{(1000 km)(x - x_j)(x - x_{j+1})}{2} | \leq 10^{-3} km
\end{equation}
\par Por lo tanto nos queda que
\begin{equation}
	\Delta t_i \leq 0.000125 h \Rightarrow | E_L^1(x) | \leq 10^{-3} km
\end{equation}
\\
\par Sabemos entonces que si tomamos intervalos de tiempo iguales a 0.000125 horas vamos a estar respetando la cota de error impuesta.


\section{Ejercicio 1}
\par Veamos
\begin{equation}
	E_L^1(x) =  \frac{f^2(\xi(x))(x - x_j)(x - x_{j+1})}{2!}
\end{equation}
\par Pero por el enunciado sabíamos que
\begin{equation}
	f^2(x) \leq (1000 \frac{km}{h^2}, 1000 \frac{km}{h^2}) \forall x \in \mathds{R}
\end{equation}
\par Entonces
\begin{equation}
	E_L^1(x) \leq \frac{(1000 \frac{km}{h^2}, 1000 \frac{km}{h^2})(x - x_j)(x - x_{j+1})}{2}
\end{equation}
\par x es un punto entre $x_j$ y $x_{j+1}$, entonces $(x - x_j) \le \Delta t$ y $(x - x_{j+1}) \le \Delta t$.
\par Entonces nos queda
\begin{equation}
	(x - x_j)(x - x_{j+1}) \leq (\Delta t_i)^2
\end{equation}
\par Volviendo nos queda que
\begin{equation}
	E_L^1(x) \leq \frac{(1000 \frac{km}{h^2}, 1000 \frac{km}{h^2})(x - x_j)(x - x_{j+1})}{2} \leq \frac{(1000 \frac{km}{h^2}, 1000 \frac{km}{h^2})(\Delta t_i)^2}{2}
\end{equation}
\begin{equation}
	E_L^1(x) \leq \frac{(1000 \frac{km}{h^2}, 1000 \frac{km}{h^2})(\Delta t_i)^2}{2}  = (\frac{1000 \frac{km}{h^2} * (\Delta t_i)^2}{2}, \frac{1000 \frac{km}{h^2} * (\Delta t_i)^2}{2})
\end{equation}
\par Queremos que el error sea menor que $10^{-3}km$. Para esto vamos a plantear la ecuación de la circunferencia con $radio = 10^{-3}km$ y veremos que valores de x e y nos sirven para manternernos dentro de la circunferencia.

\begin{equation}
	x^2 + y^2 \leq (10^{-3} km)^2
\end{equation}
\par Vamos a tomar como x e y los valores obtenidos de $E_L^1(t)$) y buscaremos que cumplan la ecuación planteada.
\begin{equation}
	(\frac{1000 \frac{km}{h^2} * (\Delta t_i)^2}{2})^2 + (\frac{1000 \frac{km}{h^2} * (\Delta t_i)^2}{2})^2 \leq (10^{-3} km)^2
\end{equation}
\begin{equation}
	\Leftrightarrow \frac{1000000 \frac{km^2}{h^4} * (\Delta t_i)^4}{4} + \frac{1000000 \frac{km^2}{h^4} * (\Delta t_i)^4}{4} \leq 10^{-6} km^2
\end{equation}
\begin{equation}
	\Leftrightarrow \frac{1000000 \frac{km^2}{h^4} * (\Delta t_i)^4}{2} \leq 10^{-6} km^2
\end{equation}
\begin{equation}
	\Leftrightarrow (\Delta t_i)^4 \leq \frac{10^{-6} km^2 * 2}{1000000 \frac{km^2}{h^4}}
\end{equation}
\begin{equation}
	\Leftrightarrow (\Delta t_i)^4 \leq 20^{-12} h^4
\end{equation}
\begin{equation}
	\Leftrightarrow | \Delta t_i | \leq 0.001 h
\end{equation}
\par Sabemos entonces que si tomamos intervalos \Delta t menores o iguale que 0.001 h estaremos respetando la cota de error impuesta. Tomaremos entonces \Delta t = 0.001 h para tener la cota superior.

\section{Ejercicio 2}


\end{document}