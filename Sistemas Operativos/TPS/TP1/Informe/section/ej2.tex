\section{Ejercicio 2}
Las tareas que ejecuta el grupo de competencia de Data Mining se pueden escribir así:\\
TaskCPU 500\\
TaskConsola 10 1 4\\
TaskConsola 20 1 4\\
TaskConsola 30 1 4\\

La primer tarea simula el algoritmo que usa intensivamente la cpu, las otras tres simulan a los tres usuarios que se conectan al servidor.
La simulación de las ejecuciones de estas tareas\\
para uno, dos y cuatro núcleos con un scheduler FCFS, son las siguientes respectivamente.\\

\begin{figure}[h]
  \centering
    \includegraphics[width=1\textwidth]{images/ej2_1core.png}
  \caption{Ejecución con un núcleo}
  \label{fig:imagen2_1}
\end{figure}
\begin{figure}[h]
  \centering
    \includegraphics[width=1\textwidth]{images/ej2_2cores.png}
  \caption{Ejecución con dos núcleos}
  \label{fig:imagen2_2}
\end{figure}
\begin{figure}[h]
  \centering
    \includegraphics[width=1\textwidth]{images/ej2_4cores.png}
  \caption{Ejecución con cuatro núcleos}
  \label{fig:imagen2_4}
\end{figure}
{\bf Latencias de las tareas:}\\\\
Con un núcleo:\\
Primera tarea: 5\\
Segunda tarea: 510\\
Tercera tarea: 550\\
Cuarta tarea: 620\\\\
Con dos núcleos:\\
Primera tarea: 5\\
Segunda tarea: 5\\
Tercera tarea: 42\\
Cuarta tarea: 112\\\\
Con cuatro núcleos:\\
Primera tarea: 5\\
Segunda tarea: 5\\
Tercera tarea: 5\\
Cuarta tarea: 5\\\\
En base a los gráficos obtenidos y los cálculos realizados, podemos concluir que First Come First Served no es una buena política de scheduling para ejecutar tareas interactivas.  

